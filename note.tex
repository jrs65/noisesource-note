%!TEX program = lualatex
\documentclass[oldfontcommands,letter,11pt,oneside,article]{memoir}

%!TEX root=regrid.tex
\usepackage{hyperref}
\usepackage{multicol}


\usepackage{fontspec}
\setmainfont[
  Ligatures=TeX,
  SmallCapsFeatures={Renderer=Basic}
]{Calendas Plus}

\usepackage{graphicx}

\usepackage[dvipsnames]{xcolor}


\definecolor{ucol}{rgb}{.6,0,0}
\definecolor{webgreen}{rgb}{0,.5,0}
\definecolor{halfgray}{gray}{0.55}
\hypersetup{%
    colorlinks=true, linktocpage=true, pdfstartpage=3, pdfstartview=FitV,%
    % uncomment the following line if you want to have black links (e.g., for printing)
    %colorlinks=false, linktocpage=false, pdfborder={0 0 0}, pdfstartpage=3, pdfstartview=FitV,%
    breaklinks=true, pdfpagemode=UseNone, pageanchor=true, pdfpagemode=UseOutlines,%
    plainpages=false, bookmarksnumbered, bookmarksopen=true, bookmarksopenlevel=1,%
    hypertexnames=true, pdfhighlight=/O,%hyperfootnotes=true,%nesting=true,%frenchlinks,%
    urlcolor=ucol, linkcolor=RoyalBlue, citecolor=webgreen, %pagecolor=RoyalBlue,%
    %urlcolor=Black, linkcolor=Black, citecolor=Black, %pagecolor=Black,%
    pdftitle={Research Statement},%
    pdfauthor={Richard Shaw},%
    pdfsubject={},%
    pdfkeywords={},%
    pdfcreator={pdfLaTeX},%
    pdfproducer={LaTeX}%
}

\usepackage{letterspace}


\newcommand{\lssecstyle}[1]{\raggedright\scshape\textls[80]{\MakeLowercase{\large\noindent #1}}}%

\setsecheadstyle{\lssecstyle}
\setbeforesecskip{-1.3\onelineskip}
\setaftersecskip{0.7\onelineskip}

\setsubsecheadstyle{\sethangfrom{\noindent ##1}\raggedright\itshape}
\setbeforesubsecskip{-1.2\onelineskip}
\setaftersubsecskip{0.6\onelineskip}


\setlrmarginsandblock{9pc}{*}{1}
\setulmarginsandblock{3pc}{5pc}{*}
\setcolsepandrule{1.5pc}{0pt}

\linespread{1.07}
% \counterwithout{section}{chapter}

% \captionnamefont{\small}\captiontitlefont{\small}

\checkandfixthelayout

\pagestyle{plain}

\renewcommand{\bibsection}{%
  %\section*{\bibname} \prebibhook}
  {\hfill\rule{0.8\linewidth}{0.7pt}\hfill} \prebibhook}
%\setlength{\bibitemsep}{-0.3\parsep}
\setlength{\bibitemsep}{0pt}

\usepackage{xspace}
\newcommand{\tcm}{21$\,$cm\xspace}
\newcommand{\tcite}[1]{[{\color{webgreen}#1}]}

\usepackage[style=numeric,backend=bibtex,maxbibnames=1,doi=false]{biblatex}
\AtEveryBibitem{\clearfield{title}\clearfield{date}\clearfield{month}}
\renewcommand{\bibfont}{\normalfont\small}
\renewbibmacro{in:}{}
\addbibresource{statement.bib}

%!TEX root = paper.tex

\usepackage{amsmath}
\usepackage{mathtools}
%\renewcommand{\la}{\ensuremath{\left\langle}}
\newcommand{\la}{\ensuremath{\left\langle}}
\newcommand{\ra}{\ensuremath{\right\rangle}}

\newcommand{\lv}{\ensuremath{\left\lvert}}
\newcommand{\rv}{\ensuremath{\right\rvert}}

\newcommand{\ls}{\ensuremath{\left[}}
\newcommand{\rs}{\ensuremath{\right]}}

\newcommand{\lp}{\ensuremath{\left(}}
\newcommand{\rp}{\ensuremath{\right)}}

\renewcommand{\vec}[1]{\boldsymbol{#1}}
\newcommand{\mat}[1]{\boldsymbol{\mathbf{#1}}}

\newcommand{\hconj}{\ensuremath{\dagger}}

\newcommand{\vx}{\vec{x}}
\newcommand{\vg}{\vec{g}}
\newcommand{\vs}{\vec{s}}
\newcommand{\vn}{\vec{n}}
\newcommand{\va}{\vec{a}}
\newcommand{\vv}{\vec{v}}
\newcommand{\vw}{\vec{w}}
\newcommand{\vk}{\vec{k}}

\newcommand{\mS}{\mat{S}}
\newcommand{\mN}{\mat{N}}
\newcommand{\mR}{\mat{R}}
\newcommand{\mC}{\mat{C}}
\newcommand{\mB}{\mat{B}}
\newcommand{\mU}{\mat{U}}
\newcommand{\mV}{\mat{V}}
\newcommand{\mE}{\mat{E}}
\newcommand{\mF}{\mat{F}}
\newcommand{\mJ}{\mat{J}}
\newcommand{\mI}{\mat{I}}


\newcommand{\mNh}{\mat{N}^{\scriptscriptstyle -\frac{1}{2}}}

\newcommand{\mQ}{\mat{Q}}
\newcommand{\mP}{\mat{P}}
\newcommand{\mQt}{\tilde{\mat{Q}}}

\newcommand{\mCt}{\tilde{\mat{C}}}
\newcommand{\mSt}{\tilde{\mat{S}}}
\newcommand{\mNt}{\tilde{\mat{N}}}

\newcommand{\msigma}{\mat{\Sigma}}
\newcommand{\mLambda}{\mat{\Lambda}}
\newcommand{\mLambdat}{\tilde{\mat{\Lambda}}}

%\newcommand{\tcm}{\ensuremath{21\,\mathrm{cm}}}

\newcommand{\vnhat}{\hat{\vec{n}}}
\newcommand{\vuhat}{\hat{\vec{u}}}
\newcommand{\vzhat}{\hat{\vec{z}}}
\newcommand{\vyhat}{\hat{\vec{y}}}
\newcommand{\vxhat}{\hat{\vec{x}}}
\newcommand{\vphat}{\hat{\vec{p}}}
\newcommand{\vdhat}{\hat{\vec{d}}}
\newcommand{\vghat}{\hat{\vec{g}}}

\newcommand{\vu}{\vec{u}}

\newcommand{\brsc}[1]{{\ensuremath{\scriptscriptstyle (#1)}}}

\newcommand{\calP}{\mathcal{P}}
\newcommand{\calN}{\mathcal{N}}
\newcommand{\calL}{\mathcal{L}}

\newcommand{\tmax}{\ensuremath{\text{max}}}

\newcommand{\mBt}{\tilde{\mat{B}}}
\newcommand{\vnt}{\tilde{\vec{n}}}
\newcommand{\vvt}{\tilde{\vec{v}}}
\newcommand{\vxt}{\tilde{\vec{x}}}

\newcommand{\mBb}{\bar{\mat{B}}}
\newcommand{\vnb}{\bar{\vec{n}}}
\newcommand{\vvb}{\bar{\vec{v}}}

\newcommand{\mCb}{\bar{\mat{C}}}
\newcommand{\mNb}{\bar{\mat{N}}}


\newcommand{\mSb}{\bar{\mat{S}}}
\newcommand{\mFb}{\bar{\mat{F}}}


\DeclareMathOperator{\sinc}{sinc}
\DeclareMathOperator{\tri}{tri}
\DeclareMathOperator{\tr}{tr}
\DeclareMathOperator{\Tr}{Tr}

\DeclareMathOperator{\Cov}{Cov}

\newcommand{\tvis}{\ensuremath{\mathcal{V}}}
\newcommand{\vis}{\ensuremath{V}}

\newcommand{\mGamma}{\mat{\Gamma}}
\newcommand{\mK}{\mat{K}}

\newcommand{\vEps}{\vec{\varepsilon}}
\newcommand{\eps}{\varepsilon}


\newcommand{\vE}{\vec{E}}
\newcommand{\vH}{\vec{H}}
\newcommand{\vS}{\vec{S}}
\newcommand{\vA}{\vec{A}}

\newcommand{\dhn}{d^2\hat{n}}

% Units
\newcommand{\hMpc}{\;h^{-1}\:\mathrm{Mpc}}
\newcommand{\ihMpc}{\;h\:\mathrm{Mpc}^{-1}}



\begin{document}

\title{Rigidisation Signal to Noise}
\author{Richard Shaw}

\maketitle

This document comes out of an email discussion about whether the accuracy with
which we can recover the gains using rigidisation is proportional to the
dynamic range\footnote{The ratio between the highest and second highest
eigenvalues.}, or the proportional to the square root. Where the reasoning
behind the latter argument is essentially that the eigenvalues represent the
power (or variance) of the gain fluctuations, and so the fluctuations
themselves are statistically proportional to the square root. This argument is
not correct however, and this document is the derivation of why, as well as a
useful description of how the instrument noise controls the dynamic range of
the solutions.

\section*{Introduction}

First let us set up the problem. What we have measured are the visibilities
during the time when the noise source is on, minus the visibilities when it's
off. This should mean that the sky signal is completely subtracted, and the only
remaining contributions are from the injected signal and a stochastic noise term
\begin{equation}
\label{eq:gain}
V_{ij} = g_i g_j^* + N_{ij} \; ,
\end{equation}
where the noise on each element $N_{ij} = n_{ij}^\text{on} - n_{ij}^\text{off}$. The noise
term is assumed to be close to Gaussian with
\begin{subequations}
\begin{align}
\label{eq:noise_stats}
\la N_{ij} \ra & = 0 \; , \\
\la N_{ij} N_{kl}^* \ra & = \calN_{(ij)(kl)}\; .
\end{align}
\end{subequations}

We can use this to construct a likelihood function in the usual way
$\calL(g) = P(V \vert G) \propto e^{-\chi^2}$, with
\begin{equation}
    \chi^2 = \sum_{ijkl} (V^*_{ij} - g^*_i g_j) \calN^{-1}_{(ij)(kl)} (V_{kl} - g_k g_l^*) \; .
\end{equation}
A good way to proceed would be to find the maximum-likelihood solution $\partial
\chi^2/\partial{g_i} = 0$, but unfortunately in the general case written above
there is no simple solution.\footnote{This is the problem of finding a Weighted
Low Rank Approximation (WLRA), see
\url{https://www.aaai.org/Papers/ICML/2003/ICML03-094.pdf} for a discussion.}
Instead lets look at a simpler case with uniform weighting. This is equivalent
to the noise-covariance being uniform and diagonal
\begin{equation}
    \label{eq:uniform_noise}
\calN_{(ij)(kl)} = \delta_{ik} \delta_{jl} \sigma^2
\end{equation}
and after minimising the likelihood leads to the constraint that
\begin{equation}
\sum_j V_{ij} g_j = \left(\sum_k g_k g_k^*\right) g_i \; \; ,
\end{equation}
or in matrix notation
\begin{equation}
    \mV \vg = \lv \vg \rv^2 \vg \; ,
\end{equation}
which is solved by an eigenvector of the matrix $\mV$ normalised such that $\lv
\vg \rv^2$ is equal to the corresponding eigenvalue $\lambda$. The
maximum-likelihood solution corresponds to picking the vector with the largest
eigenvalue.

This is the solution for the uniformly weighted case, however, we can make a
slight relaxation by a simple change of variables. If the noise covariance is of
a factorisable form
\begin{equation}
\calN_{(ij)(kl)} = \delta_{ik} \delta_{jl} \sigma_{ii} \sigma_{jj} \; ,
\end{equation}
we can make a change of variables
\begin{align}
    \label{eq:gain_transform}
    V_{ij} & \rightarrow \tilde{V}_{ij} = V_{ij} \sqrt{\frac{\sigma^2}{\sigma_{ii} \sigma_{jj}}} \notag \\
    g_i &    \rightarrow \tilde{g}_i = g_i \sqrt{\frac{\sigma}{\sigma_{ii}}}
\end{align}
to transform the problem into the uniformly weighted case. We have introduced
the arbitrary factor $\sigma$ just as a label to allow us to keep track of the
noise terms, yielding the noise covariance given in \eqref{eq:uniform_noise}.

Fortunately this factorisable case is exactly the regime we are in for an
interferometer with uncorrelated noise (requiring that $T_\text{recv} \gg
T_\text{sky}$). The variance of any element of the visibility matrix
\begin{equation}
    \sigma_{ij}^2 = \frac{T_{\text{sys},i} T_{\text{sys},j}}{\tau \Delta\nu} = \sigma_{ii} \sigma_{jj} \; .
\end{equation}
In practice this can be done simply by normalising by the geometric mean of the
auto correlations. Note that this must be the full auto-correlation before
differencing (i.e. it should be the `on' time sample).


\section*{Perturbation Theory}

We are going to solve for the gain vector in \eqref{eq:gain} by
eigendecomposition of the visibility matrix $V_{ij}$. As discussed above we have
assumed that the matrix has been rescaled such that the noise is uniform like
\eqref{eq:uniform_noise}.

In the noiseless limit ($\sigma^2 \rightarrow 0$), the system has only one
non-zero eigenvalue $\lambda_0 = g^2$, and eigenvector $\vx_0 = \vghat$, the unit
vector pointing in the direction of $\vg$ (where we have chosen the
normalisation used in \eqref{eq:gain}). The other eigenvalues $\lambda_j$ are
all zero (where $j$ always indexes $j > 0$), and the eigenvectors $\vx_j$ span
the space orthogonal to $\vx_0 = \vghat$.

To work out the effect of the noise added into the visibility matrix, we'll
treat it as a perturbation to the noiseless system. We'll denote this matrix
perturbation $\mN$ which should not be confused with the noise covariance matrix
$\calN$ used above, $\mN$ is a specific realisation from the noise. We'll need
to use eigenvector perturbation theory, which is well described on Wikipedia
\footnote{\url{http://en.wikipedia.org/wiki/Eigenvalue_perturbation}}, but may
be familiar from perturbation theory in undergraduate Quantum Mechanics! Using
this, the highest eigenvalue (which is the one we really care about)
\begin{equation}
\tilde{\lambda}_0 = g^2 + \vghat^\hconj \mN \vghat
\end{equation}
and the other eigenvalues (which were previously zero) are
\begin{equation}
\tilde{\lambda}_j = \vx_j^\hconj \mN \vx_j \; .
\end{equation}

We also care about the highest eigenvector, which using perturbation theory comes out to be
\begin{equation}
\tilde{\vx}_0 = \vghat + \frac{1}{g^2} \sum_j \vx_j ( \vx_j^\hconj \mN \vghat ) \; .
\end{equation}
We can simplify this somewhat by
\begin{align}
\tilde{\vx}_0 & = \vghat + \frac{1}{g^2} \sum_a \vx_a ( \vx_a^\hconj \mN \vghat ) - \frac{1}{g^2} \vghat ( \vghat^\hconj \mN \vghat )\\
& = \vghat + \frac{1}{g^2} (\mI - \vghat\vghat^\hconj) \mN \vghat
\end{align}
where the summation on $a$ is over all eigenvalues including zero, and we've
used the fact that the full set of eigenvectors $\vx_a$ forms an orthonormal
basis.

By inspecting these we can see that they all equations we can see that they
depend on the ratio of the instrumental noise term to the injected signal, and
that this suggests that the gain fluctuations are proportional to the dynamic
range. To make this concrete we can go through and calculate the statistics of
them. Using \eqref{eq:noise_stats} we can show that
\begin{align}
\la \tilde{\lambda}_0 \ra & = g^2 \\
\la \tilde{\lambda}_j \ra & = 0
\end{align}
and that the variances of these are
\begin{align}
\la \lambda_0^2 \ra - \la \lambda_0 \ra^2 & = \la \vghat^\hconj \mN \vghat \vghat^\hconj \mN^\hconj \vghat \ra \\
& = \sigma^2
\end{align}
and similarly
\begin{equation}
\la \lambda_j^2 \ra = \sigma^2 \; .
\end{equation}
In the work we've been doing so far, we've been defining the dynamic range as
the ratio between the first and second largest eigenvalues. Unfortunately this
is a difficult quantity to analytically express\footnote{Though it probably can
be done with random matrix theory:
\url{https://en.wikipedia.org/wiki/Random_matrix\#Spectral_theory_of_random_matrices}},
so we are going to use a slightly modified definition of the largest eigenvalue
divided by the standard deviation of the rest. Using the above, this is
\begin{equation}
    \label{eq:dynamic_range}
    r = \frac{\la \tilde{\lambda}_0 \ra}{\la \tilde{\lambda}_j^2 \ra^{1/2}} = \frac{g^2}{\sigma} \; .
\end{equation}

We are not particularly interested in what the actual eigenvector is, we are
more interested in the gain vector $\tilde{\vg}$ we would infer from the
solution
\begin{align}
    \tilde{\vg} & = \tilde{\lambda}_0^{1/2} \tilde{\vx}_0 \\
                & = \vg + \frac{1}{g} \biggl(\mI - \frac{1}{2} \vghat \vghat^\hconj \biggl) \mN \vghat \; ,
\end{align}
at first order in $\mN$. Clearly the expectation of $\tilde{\vg}$ is
\begin{equation}
\la \tilde{\vg} \ra = \vg \; ,
\end{equation}
and thus the estimate is unbiased. The covariance of the gain vector is a little more complicated
\begin{align}
\la (\tilde{\vg} - \vg) (\tilde{\vg} - \vg)^\hconj \ra_{ij} &
= \frac{1}{g^2} \la (\mI - \vghat \vghat^\hconj /2 ) \mN \vghat \vghat^\hconj \mN (\mI - \vghat \vghat^\hconj / 2) \ra_{ij} \\
& = \frac{1}{g^2} \sum_{klmn} (\delta_{ik} - \hat{g}_i \hat{g}_k^*) (\delta_{nj} - \hat{g}_n \hat{g}_j^*) \hat{g}_l \hat{g}_m^* \la N_{kl} N_{nm} \ra \; .
\end{align}
Performing all the contractions gives the gain vector covariance as
\begin{equation}
    \label{eq:gain_covariance}
\la (\tilde{\vg}_i - \vg) (\tilde{\vg}_j - \vg)^\hconj\ra = \frac{\sigma^2}{g^2} \biggl(\mI - \frac{3}{4}\vghat \vghat^\hconj \biggr) \; .
\end{equation}

\section*{Putting it all together}

Let's go back and try and relate what we've done so far back to the actual noise
source calibration procedure. First let's write down a fairly general model for
what the gain vector is
\begin{equation}
    g_i = s(t)^{1/2} \gamma_i(t) \alpha_i \; .
\end{equation}
In this $s(t)$ is the power of the injected signal, which we combined into the
gain vector in \eqref{eq:gain}. The quantity $\gamma_i(t)$ is the time variable
part of the gain for each input (the quantity we are trying to solve for), which
we assume to be $\sim 1$. Finally $\alpha_i$ is a static gain representing the
transfer function of the injected signal. The illumination that Juan has been
measuring is basically the square of this number. We assume one element of the
gain vector to be the directly digitized noise source, we'll denote it as $g_s =
s(t)^{1/2}$.

First we need to rescale our gain vectors to be the uniformly weighted case.
Using the transformation in \eqref{eq:gain_transform}, the rescaled gain vector
becomes
\begin{equation}
    \label{eq:gain_rescaled}
    \tilde{g}_i = \sqrt{\frac{\sigma s(t)}{\sigma_{ii}}} \gamma_i(t) \alpha_i \; .
\end{equation}
Due to a series of unfortunate notational choices, this $\tilde{g}$ which was
also used in the first section is actually the $g$ referred to in the second
section. Still let's ignore that for the moment and plough on. Putting this into the definition of dynamic range \eqref{eq:dynamic_range}, gives
\begin{equation}
    r = \frac{g^2}{\sigma} = \frac{s(t)}{\sigma_s} + \sum_{i \ne s} \frac{s(t)}{\sigma_{ii}} \lv \gamma_i(t) \rv^2 \lv \alpha_i \rv^2 \; .
\end{equation}
We can make this clearer by noting that for noise satisying the radiometer equation
\begin{equation}
    \sigma_{ii} = \frac{1}{\sqrt{\tau \Delta\nu}} (T_{\text{recv},i} + T_i)
\end{equation}
where $T_i$ is the temperature of the noise source seen by feed $i$ and is effectively
\begin{equation}
    T_i = \lv \gamma_i(t) \rv^2 \lv \alpha_i \rv^2 s(t) \; .
\end{equation}
Using these definitions we can define the fractional signal level of the noise source seen by feed $i$ as
\begin{equation}
    f_i = \frac{T_i}{T_{\text{recv},i} + T_i}
\end{equation}
allowing us to rewrite the dynamic range as
\begin{equation}
    r = \sqrt{\tau \Delta\nu} \left[ 1 + \sum_{i \ne s} f_i \right]
\end{equation}
For the current CHIME setup with $\Delta\nu = 0.4\,\mathrm{MHz}$ and $\tau =
20\,\mathrm{s}$, giving $s(t) / \sigma_s \sim 3000$, such that the dynamic range
$r$ should be greater than $35\,\mathrm{dB}$ (from the first term in the
parentheses). The contribution of the other inputs to the solution is suppressed
by factors of $\lv \alpha_i \rv^2 / \sigma_{ii}$. In the worst case these
contribute nothing (giving the $35\,\mathrm{dB}$ lower bound). In the best case
the noise source is bright compared to $T_\text{recv}$ for each feed, and thus
each contributes with the maximum amount it could from the radiometer equation
$f_i \sim 1$, giving $r \sim N_\text{feed} \sqrt{\tau \Delta\nu}$. For
Pathfinder parameters this gives $58\,\mathrm{dB}$, though we should note this
situation is probably undesirable as it means for each feed $T_\text{sys}$ is
dominated by the noise source.

Finally let's have a look at the expected error on the solution for each gain.
Ignoring the small correlated part of \eqref{eq:gain_covariance}, we have that
\begin{equation}
    \delta{g}_i \sim \frac{\sigma}{g} \; ,
\end{equation}
inserting the rescaled gain \eqref{eq:gain_rescaled} and rearranging for $\gamma_i$ we find
\begin{equation}
    \delta{\gamma}_i \sim \sqrt{\frac{\sigma_{ii}}{\alpha_i s(t)}} \sqrt{\frac{1}{r}} \; .
\end{equation}
Rewriting in the same way as above, we find that
\begin{equation}
    \delta{\gamma}_i \sim \frac{1}{\sqrt{\tau \Delta\nu}} \left[f_i \left(1 + \sum_{i \ne s} f_i \right) \right]^{-1/2} \; .
\end{equation}
As an example, if all the feeds were illuminated such that the injected signal
was $10 \%$ of the total power (i.e. $f_i = 0.1$), the error on the gain would
be $\sim 0.02 \%$


\end{document}
